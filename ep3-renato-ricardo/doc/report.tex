\documentclass{beamer}
\usepackage[brazilian]{babel}
\usepackage[utf8]{inputenc}
\usepackage[T1]{fontenc}
\usepackage{listings}

\mode<presentation> {

 \usetheme{Frankfurt}
 \setbeamertemplate{footline} % To remove the footer line in all slides uncomment this line

 \setbeamertemplate{footline}[page number] % To replace the footer line in all slides with a simple slide count uncomment this line

 %\setbeamertemplate{navigation symbols}{} % To remove the navigation symbols from the bottom of all slides uncomment this line
}

\usepackage{graphicx} % Allows including images
\usepackage{booktabs} % Allows the use of \toprule, \midrule and \bottomrule in tables

%----------------------------------------------------------------------------------------
% TITLE PAGE
%----------------------------------------------------------------------------------------

\title[EP3 - MAC0422]{EP3 - MAC0422 - 2015 \\
                      main, file, directory, regular, command, block e stream} % The short title appears at the bottom of every slide, the full title is only on the title page
\author{Renato Lui Geh e Ricardo Fonseca}
\date{}
\begin{document}

\begin{frame}
  \titlepage % Print the title page as the first slide
\end{frame}


%----------------------------------------------------------------------------------------
% PRESENTATION SLIDES
%----------------------------------------------------------------------------------------

%------------------------------------------------
\section{main} % Sections can be created in order to organize your presentation into discrete blocks, all sections and subsections are automatically printed in the table of contents as an overview of the talk
%------------------------------------------------

\begin{frame}
  \frametitle{Linha de input}
  Assim como nos últimos EPs, há uma string \texttt{cmd\_line} que é a linha de input do usuário. Em seguida dividimos \texttt{cmd} em tokens. \\~\\
  Isso é feito na função \texttt{Tokenize}. Os tokens são armazenados numa tabela, sendo que o primeiro elemento equivale ao próprio comando, e não o primeiro argumento do comando.
\end{frame}

\begin{frame}
  \frametitle{Prompt e Comando}
  Foi usado \texttt{readline} e \texttt{history} para o prompt, assim como foi feito no EP1.
  Com o comando do usuário e seus argumentos, passamos a tabela com tais valores para a função
  \texttt{CommandToFunction} do módulo \texttt{utils}, que realizará, se possível, a função equivalente. 
\end{frame}


%------------------------------------------------
\section{file}
%------------------------------------------------

\begin{frame}
  \frametitle{Atributos}
  A classe abstrata \texttt{file} é a representação de um arquivo em nosso programa. Ela possui os seguintes atributos:\\~\\
       \texttt{string name\_}: Nome do arquivo \\~\\
  \texttt{time\_t t\_create\_}: Tempo de criação do arquivo \\~\\
  \texttt{time\_t t\_modify\_}: Tempo da última modificação do arquivo \\~\\
  \texttt{time\_t t\_access\_}: Tempo do último acesso ao arquivo
  
  
\end{frame}

\begin{frame}
	 \frametitle{Métodos}
	Além disso, file possui um método para mudar seu nome (\texttt{Rename()}), um para que calcula quantos blocos o arquivo ocupa (\texttt{long int Block()}) e 3 métodos que atualizam os tempos de criação, modificação e último acesso:\\~\\
	\texttt{RefreshCreationTime()},\\~\\
	\texttt{RefreshModifiedTime()} e \\~\\
	\texttt{RefreshAccessedTime()}, respectivamente. \\~\\
	
	Existem também duas funções abstratas, \texttt{long int Size()} e \texttt{bool IsDirectory()}, que retornam o tamanho em bytes do arquivo e se essa classe na verdade é um diretório.
\end{frame}

\begin{frame}
	\frametitle{Operadores}
	Por último, damos override nos operadores \texttt{<, > e ==}, para comparar arquivos. Escolhemos fazer as comparações lexicográficas, para depois facilitar a impressão em ordem alfabética. 
\end{frame}

%------------------------------------------------
\section{directory}
%------------------------------------------------

\begin{frame}
  \frametitle{Atributos}
  \texttt{directory} é uma classe derivada de \texttt{file}. Seus atributos são;:\\~\\
  
  \texttt{forward\_list<File*> files\_}   : Lista dos arquivos dentro do diretório\\~\\
  \texttt{int n\_files\_}                : Número de arquivos dentro do diretório\\~\\
  \texttt{long int files\_sizeb\_}       : Tamanho em bytes dos arquivos dentro do diretorio\\~\\
      

\end{frame}

\begin{frame}
	\frametitle{Métodos}
	A classe possue os seguintes métodos:
	
	\texttt{ListFiles(FILE *stream)}     : Imprime em stream os arquivos dentro do diretório. \\~\\
	\texttt{InsertFile(File *f)}         : Insere o arquivo \texttt{f} nesse diretório.\\~\\
	\texttt{RemoveFile(string &name)}    : Remove o arquivo com nome \texttt{name} deste diretório.\\~\\
	\texttt{File* FindFile(string &name)}: Procura e retorna o arquivo de nome \texttt{name}.\\~\\ 
	
	
	Além disso foram implementadas os métodos \texttt{long int Size} e \texttt{bool IsDirectory} da classe \texttt{file}. 
\end{frame}


%------------------------------------------------
\section{regular}
%------------------------------------------------

\begin{frame}
  \frametitle{Arquivo Regular}
	\texttt{regular} também é uma classe derivada de file, que representa um arquivo de puro texto.
  Possui como atributo \texttt{long int sizeb\_}, que guarda o tamanho em bytes do arquivo, e métodos para ler e escrever seu conteúdo: \texttt{string ReadContent(FILE *stream) e WriteContent(string data)}.\\~\\

  Além disso \texttt{regular} da override nas funções de \texttt{Size e isDirectory} para retornar o tamanho do arquivo e falso, respectivamente.
\end{frame}


%------------------------------------------------
\section{command}
%------------------------------------------------

\begin{frame}
  \frametitle{Comandos}
  O módulo \texttt{command} possui um namespace com todos os commandos pedidos no enunciado, além de \texttt{SetPath(Directory &dir)} que serve para setar o caminho atual do usuário.
\end{frame}

%------------------------------------------------
\section{block}
%------------------------------------------------

\begin{frame}
  \frametitle{Blocos}
	Nosso módulo \texttt{block} tem a implementação da classe de mesmo nome pra representar um bloco de espaço no disco.
\end{frame}

\begin{frame}
  \frametitle{Atributos}
  Ele possui três atributos:\\~\\
  
  \texttt{long int next\_} : Índice do próximo bloco do arquivo. \\~\\
  \texttt{long int prev\_} : Índice do bloco anterior do arquivo.\\~\\
  \texttt{long int index\_}: Índice desse bloco.\\~\\
  \texttt{string content\_}: Conteúdo em texto do bloco.
\end{frame}

\begin{frame}
  \frametitle{Métodos}
  A classe possui métodos para setar e retornar os índices (atuais quanto próximos/anteriores) e tamanho em bytes do conteúdo.\\~\\
  Além disso ele possui duas funções para escrever data:\\~\\

  \texttt{Write(string &data)} : Sobrescreve o arquivo com \texttt{data}.\\~\\
  \texttt{Append(string &data)}: Escreve no fim do arquivo \texttt{data}. 
\end{frame}

%------------------------------------------------
\section{stream}
%------------------------------------------------

\begin{frame}
  \frametitle{Stream}
  Esse módulo possui um namespace que organiza as entradas e saídas do programa.
\end{frame}

\begin{frame}
  \frametitle{Output}
	Namespace \texttt{Output} lida com as saidas para o sistema de arquivo com 4 métodos:\\~\\

  \texttt{Open(string &filename)}: Abre um arquivo\\~\\
  \texttt{WriteMeta()}: Escreve a metadata do sistema de arquivos (o bitmap dos espaços livres, informações do FAT e Root). \\~\\
  \texttt{Write(Block *head)}: Escreve as informações de todos os blocos no FAT. \\~\\
  \texttt{Close()}: Fecha o arquivo.

\end{frame}

\begin{frame}
  \frametitle{Input}
  Namespace \texttt{Input} lida com as entradas do usuário e programa com 4 métodos:\\~\\
  
  \texttt{Open(string &filename)}: Abre um arquivo\\~\\
  \texttt{ReadMeta()}: Lê a metadata do sistema de arquivos. \\~\\
  \texttt{Block* Read(long int index)}: Lê apenas o conteúdo de um bloco. \\~\\
  \texttt{Close()}: Fecha o arquivo.
\end{frame}

\begin{frame}
  \frametitle{Metadata}
  Namespace que tem atributos em pares (com a posição e tamanho) de metadata do sistema de arquivos:\\~\\

  \texttt{pair<long int, long int> kBitmapBlock}: Bitmap das posições livres de blocos (1 para ocupado, 0 para livre).\\~\\
  \texttt{pair<long int, long int> kFatBlock}: Bloco de Informações do FAT. \\~\\
  \texttt{pair<long int, long int> kRootBlock}: Bloco de informações do Root do sistema de arquivos.\\~\\
  \texttt{pair<long int, long int> kDiskBlock}: Blocos do restante do disco.
\end{frame}

\begin{frame}
  \frametitle{Exception}
  Por último, o namespace \texttt{Exception} serve para mensagens de erro, caso tenha um arquivo inválido ou algo tente ultrapassar o limite de 100MB do sistema de arquivos.
\end{frame}

%------------------------------------------------
\section{Observações}
%------------------------------------------------

\begin{frame}
  \frametitle{Observações Finais}
  Colocar aqui outras funções importantes que não mencionei, caso existam.
\end{frame}

%------------------------------------------------
\section{Observações}
%------------------------------------------------

\begin{frame}
  \frametitle{Observações}
  Entretanto, por falta de organização, muitas partes do EP ficaram incompletas. Percebemos que não era para ter coisas nas memórias, como
  era o caso das calsses File, Directory e Regular, e no fim todas foram removidas, e o módulo Stream lida direto com as informações do sistema de arquivos.
\end{frame}

%------------------------------------------------

\end{document}
