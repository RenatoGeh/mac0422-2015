\documentclass{beamer}
\usepackage[brazilian]{babel}
\usepackage[utf8]{inputenc}
\usepackage[T1]{fontenc}
\usepackage{listings}

\mode<presentation> {

 \usetheme{Frankfurt}
 \setbeamertemplate{footline} % To remove the footer line in all slides uncomment this line

 \setbeamertemplate{footline}[page number] % To replace the footer line in all slides with a simple slide count uncomment this line

 %\setbeamertemplate{navigation symbols}{} % To remove the navigation symbols from the bottom of all slides uncomment this line
}

\usepackage{graphicx} % Allows including images
\usepackage{booktabs} % Allows the use of \toprule, \midrule and \bottomrule in tables

%----------------------------------------------------------------------------------------
% TITLE PAGE
%----------------------------------------------------------------------------------------

\title[EP2 - MAC0422]{EP2 - MAC0422 - 2015 \\
                      main, utils,  mem\_mgr e page\_mgr} % The short title appears at the bottom of every slide, the full title is only on the title page
\author{Renato Lui Geh e Ricardo Fonseca}
\date{}
\begin{document}

\begin{frame}
  \titlepage % Print the title page as the first slide
\end{frame}


%----------------------------------------------------------------------------------------
% PRESENTATION SLIDES
%----------------------------------------------------------------------------------------

%------------------------------------------------
\section{main} % Sections can be created in order to organize your presentation into discrete blocks, all sections and subsections are automatically printed in the table of contents as an overview of the talk
%------------------------------------------------

\begin{frame}
  \frametitle{Linha de input}
  Assim como no EP1, há uma string \texttt{cmd} que é a linha de input do usuário. Em seguida
  dividimos \texttt{cmd} em tokens tais que cada par de token $(t_i,t_{i+1})$, quando concatenados
  produzem uma subsequência de \texttt{cmd}. Ou seja, a concatenação $(t_0,...,t_k)$ resulta em
  \texttt{cmd}. Isso é feito na função \texttt{extract\_args}. O token $t_i$ equivale ao código
  \texttt{args\_table[i]}. Note que $t_0$ equivale ao próprio comando, e não o primeiro argumento
  do comando.
\end{frame}

\begin{frame}
  \frametitle{Prompt}
  Foi usado \texttt{readline} para o prompt, assim como foi feito no EP1.
\end{frame}

\begin{frame}[fragile]
  \frametitle{\texttt{evaluate\_str}}
  C e C++ não permitem usar \texttt{switch} com cstrings (\texttt{char*}). Portanto para podermos
  comparar o comando num \texttt{switch} precisamos transformar a string em um \texttt{unsigned
  int}. Para isso usamos \texttt{evaluate\_str}.

  \begin{lstlisting}[language=C++,basicstyle=\scriptsize]
/* Transforma string em um unsigned integer. */
constexpr unsigned int evaluate_str(const char* str, int h = 0) {
  return !str[h] ? 5381 : (evaluate_str(str, h+1)*33) ^ str[h];
}
  \end{lstlisting}

  O que fazemos é passar por cada caractere $c_i$ da string e transforma-lo em um número inteiro
  não negativo $k_i$ tal que ele seja dependente do último caractere, ou seja, dada uma
  transformação $T:char \to \mathbb{Z}^+$ então $k_i = T(c_i)|T(c_{i-1})$ (leia-se $k_i$ é a
  transformação de $c_i$ dada a transformação de $c_{i-1}$).
\end{frame}

\begin{frame}
  \frametitle{Parsing}
  A função \texttt{parse} lê o arquivo de trace e cria os processos e memória a serem simuladas.
\end{frame}

%------------------------------------------------
\section{utils}
%------------------------------------------------

\begin{frame}[fragile]
  \frametitle{Escrevendo bytes}
  Para escrevermos a memória byte a byte, foi usada a função \texttt{fwrite} da \texttt{cstdio}.
  Para garantirmos que o que realmente escrevemos é um byte, usamos um \texttt{unsigned char}. Para
  representar o número -1, foi escolhido 255 para representar memória livre. Portanto, no total,
  podemos ter 255 processos rodando.

  \begin{lstlisting}[language=C++,basicstyle=\scriptsize]
void __write_bytes(FILE* stream, int i, int f, unsigned char val) {
  fseek(stream, i, SEEK_SET);

  while (i++ <= f)
    fwrite(&val, sizeof(val), sizeof(val), stream);
}
  \end{lstlisting}

  Aqui pode-se ver que o intervalo $[i, f]$ (ambos extremos inclusivos) é escrito com o valor $val$.
\end{frame}

%------------------------------------------------
\section{mem\_mgr}
%------------------------------------------------

\begin{frame}
  \frametitle{Gerenciador de Memória}
  Para facilitar o trabalho da implementação do nosso Gerenciador de Memória, criamos dois arquivos: \texttt{utils.c} e \texttt{utils.h}. Nesses arquivos foram criados as structs para representar \texttt{mem\_node}, \texttt{size\_node}, funções para auxiliar algoritmos e escrever bytes no arquivo de saída. \\~\\

  Listas duplamente ligadas com cabeça foram criadas para armazenar e manipular os blocos de memória mais facilmente. Para o algortimo Quick Fit usamos uma \texttt{lista de listas} para separar os diferentes tamanhos de espaços livres na memória.
\end{frame}

%------------------------------------------------

\begin{frame}
  \frametitle{mem\_node e size\_node}
  Para manipular a memória virtual, utilizamos duas structs diferentes, uma representando um bloco na memória virtual e outro que é apenas usado para o algoritmo Quick Fit \\~\\

  \texttt{mem\_node}: Possui 6 campos: (\texttt{char t}) para o tipo do bloco, (\texttt{int i}) para a posição início da memória do bloco, (\texttt{int s}) para tamanho do bloco, (\texttt{mem\_node *n}) e (\texttt{mem\_node *p}) como ponteiros para os pŕoximos nós da lista. \\~\\

  \texttt{size\_node}: Lista de listas de tamanhos de blocos para o algoritmo Quick Fit. Possui 4 campos. (\texttt{int s}) para tamanho dos blocos da lista que o nó aponta, (\texttt{mem\_node *f}) como ponteiro para uma lista ligada sem cabeça com os blocos livres de tamanho \texttt{s}, (\texttt{size\_node *n}) e (\texttt{size\_node *p}) como ponteiros para os pŕoximos nós da lista.
\end{frame}

%------------------------------------------------

\begin{frame}
  \frametitle{Algoritmos}
  Para ser mais fácil de se escolher qual gerenciador usar, criamos a variável \texttt{manager}, que é um ponteiro para função. Os argumentos de linha de comando são lidos e a função é atribuída a \texttt{manager} em seguida. \\~\\

  O gerenciador pelo método Quick Fit utiliza uma lista de listas, que é criado apenas no caso dele ser escolhido ao rodar o programa.
\end{frame}

%------------------------------------------------

\begin{frame}
  \frametitle{Listas Ligadas}
  Em todos os gerenciadores implementados foram usadas a lista com a cabeça \texttt{v\_mem\_h}. Para a memória física total utiliza-se \texttt{t\_mem\_h} \\~\\

  \texttt{v\_mem\_h}: guarda todos os blocos de memória livres ou ocupados na memória virtual. O algoritmo Quick Fit utiliza de forma um pouco diferente essa lista. \\~\\

  \texttt{t\_mem\_h}: guarda todos os blocos de memória ocupando a memória física total.
\end{frame}

%------------------------------------------------

\begin{frame}
  \frametitle{First Fit (FF)}
  Este gerenciador usa diretamente a lista da memória virtual. Quando o \texttt{t\_secs} chega no instante \texttt{t0} de um processo, se há algum espaço livre que ele caiba, o processo é atribuído àquela parte da memória virtual.
  O bloco de memória é mudado para \texttt{P}. \\~\\

  O processo permanece na memória até terminar.
\end{frame}

%------------------------------------------------

\begin{frame}
  \frametitle{Next Fit (NF)}
  Este gerenciador também usa diretamente a lista da memória virtual. Quando o \texttt{t\_secs} chega no instante \texttt{t0} de um processo, se há algum espaço livre que caiba o processo, o processo é atribuído àquela parte da memória virtual. Porém usamos um apontador de nó especial \texttt{v\_last} que marca a última posição vista na lista, dessa forma para futuros usos da função, ela começa do último bloco criado. \\~\\

\end{frame}

%------------------------------------------------
\begin{frame}
  \frametitle{Quick Fit (QF)}
  Ao escolher o algoritmo Quick Fit, o dicionário é criado dinamicamente por nosso algoritmo, que faz o seguinte:
  \begin{itemize}
    \item Pega o valor disponível (inicialmente toda a memória virtual disponível).
    \item Divide em 4 esse espaço, e aproxima até o maior multiplo de 2 menor que esse valor.
    \item Se o resultado for maior que o limite inferior (definido por nós como 16, de acordo com o tamanho das páginas), separa 3/4 do tamanho total recebido pela função como espaço livre nas listas (para cada 1/3 é criado um bloco de espaço livre desse tamanho pŕoximo de 1/4 do tamanho total). Em seguida chama a função recursivamente para o restante do espaço.
    \item Caso o resultado seja igual ou menor que o limite inferior, separa o maior numero de intervalos com tamanho limite inferior como espaço livre na lista do quick fit.
  \end{itemize}
\end{frame}

%------------------------------------------------

\begin{frame}
  \frametitle{QF - Continuação}
  Quando o \texttt{t\_secs} chega no instante \texttt{t0} de um processo, ele procura o menor bloco livre que seja maior ou igual que o tamanho necessário do processo (percorre a lista de size\_node), e depois marca como ocupado esse bloco, e põe no começo da lista \texttt{v\_mem\_h} esse processo (essa lista ligada não possue ordenação para processos, apenas tem todos os blocos que estão sendo ocupados por processos.\\~\\

\end{frame}

%------------------------------------------------

\begin{frame}
  \frametitle{ff\_nf\_free}
  Tanto para o gerenciador de memória FF quanto NF utilizamos a mesma função para quando um processo termina. De forma simples ela transforma o bloco como livre, e com a ajuda das listas duplamente ligadas,
  verifica se o bloco estava entre algum outro bloco livre, eficientemente unindo-os em um só bloco pronto para ocupar outro processo.\\~\\

\end{frame}

%------------------------------------------------

\begin{frame}
  \frametitle{qf\_free}
  Para o gerenciador de memória QF utilizamos a função \texttt{qf\_free} para quando um processo termina. Ela transforma o bloco do processo como livre, e utilizando nossa \texttt{lista de listas}, disponibiliza como livre o bloco na sua respectiva lista (a que possue outros blocos com seu mesmo tamanho).\\~\\

\end{frame}

%------------------------------------------------
\section{page\_mgr}
%------------------------------------------------

\begin{frame}
  \frametitle{Paginação}
  Para paginação, rodamos cada processo e se no tempo $t_i$ houver um acesso, vemos se a memória a
  ser acessada está na memória física. Se não estiver, page fault e usamos um dos algoritmos de
  substituição de página. Senão não fazemos nada, já que estamos apenas simulando e não realmente
  lendo a memória.
\end{frame}

%------------------------------------------------

\begin{frame}
  \frametitle{Not Recently Used Page (NRU)}
  Foi usado um vetor \texttt{virt\_refs::internal} para representar cada página da memória virtual.
  Nesse vetor estão os dados de acesso de cada página. Dado um elemento $e_i$, se $e_i=-1$, então
  ele não foi acessado recentemente. Senão, se $e_i=0$, então ele foi acessado neste ciclo. Senão,
  entao $e_i>0$ e portanto ele foi acessado $e_i$ segundos atrás. Se $e_i$ for maior ou igual ao
  limite \texttt{INTERRUPT\_DT} que representa o tempo decorrido após cada \textit{reset}, então
  $e_i=-1$.

  A função \texttt{nru\_repl} procura por $e_i,...,e_j$ que sejam, juntos, do tamanho requisitado e
  que tenham menor peso. Definimos peso como:

  $w_i = 1$, se $e_i \geq 0$ \\
  $w_i = 0$, se $e_i < 0$

  Portanto, queremos $min(\sum_{k=i}^j w_k)$.
\end{frame}

%------------------------------------------------

\begin{frame}
  \frametitle{First-In, First-Out (FIFO)}
  Neste algoritmo de substituição de página, apenas temos uma fila \texttt{page\_queue} e toda vez
  que há um \textit{page fault}, pegam-se as $n$ primeiras páginas ocupadas (e portanto mais
  "velhas") tal que a soma dos tamanhos delas sejam a menor soma possível e que sejam também maior
  ou igual ao tamanho requerido.
\end{frame}

%------------------------------------------------

\begin{frame}
\frametitle{Second-Chance Page (SC)}
Coisas ddo amor
\end{frame}

%------------------------------------------------

\begin{frame}
\frametitle{Least Recently Used Page (LRU)}
Where is the Love
\end{frame}

%------------------------------------------------
\section{Testes e Observações Finais}
%------------------------------------------------

\begin{frame}
  \frametitle{Testes}
  Veremos alguns testes feitos com os gerenciadores de memória e algoritmos de substituição de página.
\end{frame}

%------------------------------------------------

\begin{frame}
  \frametitle{Observações}
  Nosso algoritmo Quick Fit divide a memória virtual até tamanhos maiores ou iguais que o limite inferior, o que gera um problema de desperdício de memória, pois se a memória não for múltiplo dele (16 no caso),
  é possível que até 15 bytes sejam desperdiçados e nunca utilizados, mas comparado com a memória virtual disponível não é de grande impacto esse valor.
\end{frame}

\end{document}
